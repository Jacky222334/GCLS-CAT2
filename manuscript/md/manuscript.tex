% Options for packages loaded elsewhere
\PassOptionsToPackage{unicode}{hyperref}
\PassOptionsToPackage{hyphens}{url}
\documentclass[
  english,
  man]{apa6}
\usepackage{xcolor}
\usepackage{amsmath,amssymb}
\setcounter{secnumdepth}{-\maxdimen} % remove section numbering
\usepackage{iftex}
\ifPDFTeX
  \usepackage[T1]{fontenc}
  \usepackage[utf8]{inputenc}
  \usepackage{textcomp} % provide euro and other symbols
\else % if luatex or xetex
  \usepackage{unicode-math} % this also loads fontspec
  \defaultfontfeatures{Scale=MatchLowercase}
  \defaultfontfeatures[\rmfamily]{Ligatures=TeX,Scale=1}
\fi
\usepackage{lmodern}
\ifPDFTeX\else
  % xetex/luatex font selection
\fi
% Use upquote if available, for straight quotes in verbatim environments
\IfFileExists{upquote.sty}{\usepackage{upquote}}{}
\IfFileExists{microtype.sty}{% use microtype if available
  \usepackage[]{microtype}
  \UseMicrotypeSet[protrusion]{basicmath} % disable protrusion for tt fonts
}{}
\makeatletter
\@ifundefined{KOMAClassName}{% if non-KOMA class
  \IfFileExists{parskip.sty}{%
    \usepackage{parskip}
  }{% else
    \setlength{\parindent}{0pt}
    \setlength{\parskip}{6pt plus 2pt minus 1pt}}
}{% if KOMA class
  \KOMAoptions{parskip=half}}
\makeatother
% Make \paragraph and \subparagraph free-standing
\makeatletter
\ifx\paragraph\undefined\else
  \let\oldparagraph\paragraph
  \renewcommand{\paragraph}{
    \@ifstar
      \xxxParagraphStar
      \xxxParagraphNoStar
  }
  \newcommand{\xxxParagraphStar}[1]{\oldparagraph*{#1}\mbox{}}
  \newcommand{\xxxParagraphNoStar}[1]{\oldparagraph{#1}\mbox{}}
\fi
\ifx\subparagraph\undefined\else
  \let\oldsubparagraph\subparagraph
  \renewcommand{\subparagraph}{
    \@ifstar
      \xxxSubParagraphStar
      \xxxSubParagraphNoStar
  }
  \newcommand{\xxxSubParagraphStar}[1]{\oldsubparagraph*{#1}\mbox{}}
  \newcommand{\xxxSubParagraphNoStar}[1]{\oldsubparagraph{#1}\mbox{}}
\fi
\makeatother
\usepackage{graphicx}
\makeatletter
\newsavebox\pandoc@box
\newcommand*\pandocbounded[1]{% scales image to fit in text height/width
  \sbox\pandoc@box{#1}%
  \Gscale@div\@tempa{\textheight}{\dimexpr\ht\pandoc@box+\dp\pandoc@box\relax}%
  \Gscale@div\@tempb{\linewidth}{\wd\pandoc@box}%
  \ifdim\@tempb\p@<\@tempa\p@\let\@tempa\@tempb\fi% select the smaller of both
  \ifdim\@tempa\p@<\p@\scalebox{\@tempa}{\usebox\pandoc@box}%
  \else\usebox{\pandoc@box}%
  \fi%
}
% Set default figure placement to htbp
\def\fps@figure{htbp}
\makeatother
\ifLuaTeX
\usepackage[bidi=basic,shorthands=off,]{babel}
\else
\usepackage[bidi=default,shorthands=off,]{babel}
\fi
\ifLuaTeX
  \usepackage{selnolig} % disable illegal ligatures
\fi
\setlength{\emergencystretch}{3em} % prevent overfull lines
\providecommand{\tightlist}{%
  \setlength{\itemsep}{0pt}\setlength{\parskip}{0pt}}
\usepackage[]{natbib}
\bibliographystyle{plainnat}
% Essential packages
\usepackage[utf8]{inputenc}
\usepackage[T1]{fontenc}
\usepackage{graphicx}
\usepackage{booktabs}
\usepackage{caption}
\usepackage{subcaption}
\usepackage{float}
\usepackage{csquotes}

% Document metadata
\leftheader{Running head: GCLS GERMAN VALIDATION} 
% Manuscript styling
\usepackage{upgreek}
\captionsetup{font=singlespacing,justification=justified}

% Table formatting
\usepackage{longtable}
\usepackage{lscape}
% \usepackage[counterclockwise]{rotating}   % Landscape page setup for large tables
\usepackage{multirow}		% Table styling
\usepackage{tabularx}		% Control Column width
\usepackage[flushleft]{threeparttable}	% Allows for three part tables with a specified notes section
\usepackage{threeparttablex}            % Lets threeparttable work with longtable

% Create new environments so endfloat can handle them
% \newenvironment{ltable}
%   {\begin{landscape}\centering\begin{threeparttable}}
%   {\end{threeparttable}\end{landscape}}
\newenvironment{lltable}{\begin{landscape}\centering\begin{ThreePartTable}}{\end{ThreePartTable}\end{landscape}}

% Enables adjusting longtable caption width to table width
% Solution found at http://golatex.de/longtable-mit-caption-so-breit-wie-die-tabelle-t15767.html
\makeatletter
\newcommand\LastLTentrywidth{1em}
\newlength\longtablewidth
\setlength{\longtablewidth}{1in}
\newcommand{\getlongtablewidth}{\begingroup \ifcsname LT@\roman{LT@tables}\endcsname \global\longtablewidth=0pt \renewcommand{\LT@entry}[2]{\global\advance\longtablewidth by ##2\relax\gdef\LastLTentrywidth{##2}}\@nameuse{LT@\roman{LT@tables}} \fi \endgroup}

% \setlength{\parindent}{0.5in}
% \setlength{\parskip}{0pt plus 0pt minus 0pt}

% Overwrite redefinition of paragraph and subparagraph by the default LaTeX template
% See https://github.com/crsh/papaja/issues/292
\makeatletter
\renewcommand{\paragraph}{\@startsection{paragraph}{4}{\parindent}%
  {0\baselineskip \@plus 0.2ex \@minus 0.2ex}%
  {-1em}%
  {\normalfont\normalsize\bfseries\itshape\typesectitle}}

\renewcommand{\subparagraph}[1]{\@startsection{subparagraph}{5}{1em}%
  {0\baselineskip \@plus 0.2ex \@minus 0.2ex}%
  {-\z@\relax}%
  {\normalfont\normalsize\itshape\hspace{\parindent}{#1}\textit{\addperi}}{\relax}}
\makeatother

\makeatletter
\usepackage{etoolbox}
\patchcmd{\maketitle}
  {\section{\normalfont\normalsize\abstractname}}
  {\section*{\normalfont\normalsize\abstractname}}
  {}{\typeout{Failed to patch abstract.}}
\patchcmd{\maketitle}
  {\section{\protect\normalfont{\@title}}}
  {\section*{\protect\normalfont{\@title}}}
  {}{\typeout{Failed to patch title.}}
\makeatother

\usepackage{xpatch}
\makeatletter
\xapptocmd\appendix
  {\xapptocmd\section
    {\addcontentsline{toc}{section}{\appendixname\ifoneappendix\else~\theappendix\fi: #1}}
    {}{\InnerPatchFailed}%
  }
{}{\PatchFailed}
\makeatother
\keywords{gender congruence; life satisfaction; scale validation; transgender health; psychometrics
}
\usepackage{csquotes}
\usepackage{bookmark}
\IfFileExists{xurl.sty}{\usepackage{xurl}}{} % add URL line breaks if available
\urlstyle{same}
\hypersetup{
  pdftitle={Validation of the German Version of the Gender Congruence and Life Satisfaction Scale (G-GCLS)},
  pdfauthor={Jan Ben SchulzeDepartment of Consultation-Liaison Psychiatry and Psychosomatic Medicine, University Hospital Zurich, Flavio AmmannDepartment of Consultation-Liaison Psychiatry and Psychosomatic Medicine, University Hospital Zurich, Bethany A. JonesNottingham Centre for Transgender HealthSchool of Sport, Exercise and Health Sciences, Loughborough University, Roland von KänelDepartment of Consultation-Liaison Psychiatry and Psychosomatic Medicine, University Hospital Zurich, \& Sebastian EulerDepartment of Consultation-Liaison Psychiatry and Psychosomatic Medicine, University Hospital Zurich},
  pdflang={en-EN},
  pdfkeywords={gender congruence; life satisfaction; scale validation; transgender health; psychometrics},
  hidelinks,
  pdfcreator={LaTeX via pandoc}}

\title{Validation of the German Version of the Gender Congruence and Life Satisfaction Scale (G-GCLS)}
\author{Jan Ben Schulze\textsuperscript{Department of Consultation-Liaison Psychiatry and Psychosomatic Medicine, University Hospital Zurich}, Flavio Ammann\textsuperscript{Department of Consultation-Liaison Psychiatry and Psychosomatic Medicine, University Hospital Zurich}, Bethany A. Jones\textsuperscript{Nottingham Centre for Transgender HealthSchool of Sport, Exercise and Health Sciences, Loughborough University}, Roland von Känel\textsuperscript{Department of Consultation-Liaison Psychiatry and Psychosomatic Medicine, University Hospital Zurich}, \& Sebastian Euler\textsuperscript{Department of Consultation-Liaison Psychiatry and Psychosomatic Medicine, University Hospital Zurich}}
\date{May 20, 2025}


\shorttitle{SHORT TITLE}

\authornote{

Correspondence concerning this article should be addressed to Jan Ben Schulze. E-mail: \href{mailto:jan.schulze@usz.ch}{\nolinkurl{jan.schulze@usz.ch}}

}

\affiliation{\vspace{0.5cm}\textsuperscript{uzh} Department of Consultation-Liaison Psychiatry and Psychosomatic Medicine, University Hospital Zurich\\\textsuperscript{nct} Nottingham Centre for Transgender Health\\\textsuperscript{lboro} School of Sport, Exercise and Health Sciences, Loughborough University}

\abstract{%
The Gender Congruence and Life Satisfaction Scale (GCLS) is a validated measure assessing outcomes in transgender and gender diverse individuals. This study presents the translation and validation of the German version (G-GCLS). Using a sample of 293 participants, we conducted exploratory factor analysis and assessed psychometric properties. The G-GCLS demonstrated excellent internal consistency (Cronbach's α ranging from .78 to .90) and replicated the seven-factor structure of the original scale. Factor analysis revealed strong correspondence with the original English version, particularly for the Chest, Genitalia, and Social Gender Role Recognition subscales. The results support the G-GCLS as a reliable and valid instrument for assessing gender congruence and life satisfaction in German-speaking transgender and gender diverse populations.
}



\begin{document}
\maketitle

\section{Introduction}\label{introduction}

The provision of healthcare for transgender individuals within the German-speaking healthcare system is confronting existential challenges. Transgender people experience a fundamental incongruence between the sex assigned at birth and their lived gender identity, often identifying as gender-neutral, non-gender, or gender-queer \citep{arcelus2017, richards2016}. This discrepancy is frequently accompanied by significant distress, necessitating specialized and multidisciplinary treatment approaches \citep{beek2015}.

\subsection{Current Measures}\label{current-measures}

Established measurement instruments for assessing outcomes in transgender healthcare show significant limitations. The Utrecht Gender Dysphoria Scale \citep{cohen1997} and Hamburg Body Drawing Scale \citep{becker2016} rely on binary gender concepts and focus on single aspects of gender experience. This narrow approach fails to capture the complex needs of transgender individuals \citep{bouman2017} and imposes significant respondent burden \citep{rolstad2011}.

\subsection{Present Study}\label{present-study}

The Gender Congruence and Life Satisfaction Scale (GCLS; \citet{jones2019}) addresses these limitations through a comprehensive, non-binary approach. This study validates the German version (G-GCLS) to enable its use in German-speaking contexts.

\section{Method}\label{method}

\subsection{Participants}\label{participants}

\begin{itemize}
\tightlist
\item
  Sample: 293 German-speaking transgender and gender diverse individuals
\item
  Recruitment: Gender identity clinics and community organizations
\item
  Ethics: Approved by Cantonal Ethics Committee Zurich (BASEC No.~Req-2022-00630)
\end{itemize}

\subsection{Measures}\label{measures}

The G-GCLS comprises 38 items across seven subscales:
- Genitalia (GEN): 6 items
- Chest (CH): 4 items
- Secondary Sex Characteristics (SSC): 3 items
- Social Gender Role Recognition (SGR): 4 items
- Physical/Emotional Intimacy (PEI): 4 items
- Psychological Functioning (PF): 10 items
- Life Satisfaction (LS): 7 items

Response format: 5-point Likert scale (1 = \emph{always} to 5 = \emph{never})

\subsection{Procedure}\label{procedure}

Translation process following \citet{beaton2000}:
1. Forward translation (2 translators)
2. Synthesis
3. Back-translation (2 translators)
4. Expert review
5. Pilot testing
6. Final revision

\subsection{Statistical Analysis}\label{statistical-analysis}

\begin{enumerate}
\def\labelenumi{\arabic{enumi}.}
\tightlist
\item
  Preliminary Analysis

  \begin{itemize}
  \tightlist
  \item
    KMO test
  \item
    Bartlett's test
  \item
    Item distributions
  \end{itemize}
\item
  Factor Analysis

  \begin{itemize}
  \tightlist
  \item
    Principal axis factoring
  \item
    Varimax rotation
  \item
    Loading criterion: ≥ 0.40
  \end{itemize}
\item
  Reliability Assessment

  \begin{itemize}
  \tightlist
  \item
    Internal consistency
  \item
    Item-total correlations
  \end{itemize}
\end{enumerate}

\section{Results}\label{results}

\subsection{Preliminary Analyses}\label{preliminary-analyses}

\begin{itemize}
\tightlist
\item
  KMO = 0.920 (excellent sampling adequacy)
\item
  Significant Bartlett's test (p \textless{} .001)
\item
  Normal item distributions
\end{itemize}

\subsection{Factor Analysis}\label{factor-analysis}

Seven-factor solution (57.1\% variance explained):
1. Psychological Functioning/Life Satisfaction (19.1\%)
2. Genitalia (10.5\%)
3. Chest (8.2\%)
4. Social Gender Role (6.9\%)
5. Secondary Sex Characteristics (4.8\%)
6. Physical/Emotional Aspects (4.0\%)
7. Residual Factor (3.7\%)

\subsection{Reliability}\label{reliability}

Internal consistency:
- GEN: α = 0.883
- CH: α = 0.900
- SSC: α = 0.795
- SGR: α = 0.842
- PEI: α = 0.779
- PF: α = 0.884
- LS: α = 0.811

\subsection{Validity}\label{validity}

\section{Discussion}\label{discussion}

\subsection{Factor Structure}\label{factor-structure}

The German version of the GCLS demonstrates strong psychometric properties and replicates the factor structure of the original English version. The high internal consistency values (α \textgreater{} 0.77 for all subscales) indicate reliable measurement of the intended constructs.

\subsection{Reliability and Validity}\label{reliability-and-validity}

{[}Discussion of reliability and validity findings{]}

\subsection{Strengths and Limitations}\label{strengths-and-limitations}

\subsubsection{Strengths}\label{strengths}

\begin{itemize}
\tightlist
\item
  Large validation sample
\item
  Comprehensive psychometric evaluation
\item
  Strong reliability coefficients
\item
  Clear factor structure
\end{itemize}

\subsubsection{Limitations}\label{limitations}

\begin{itemize}
\tightlist
\item
  Cross-sectional design
\item
  Lack of test-retest reliability data
\item
  Need for confirmatory factor analysis in an independent sample
\end{itemize}

\subsection{Clinical Implications}\label{clinical-implications}

The results support the use of the German GCLS in clinical practice and research with German-speaking transgender and gender diverse populations. The robust psychometric properties and clear factor structure suggest that the instrument can effectively assess gender congruence and life satisfaction in this population.

\subsection{Future Research Directions}\label{future-research-directions}

Future research should focus on:

\begin{enumerate}
\def\labelenumi{\arabic{enumi}.}
\tightlist
\item
  Confirmatory factor analysis in independent samples
\item
  Test-retest reliability assessment
\item
  Measurement invariance testing across diverse gender groups
\item
  External validity evaluation in clinical settings
\end{enumerate}

\section{Conclusion}\label{conclusion}

The German version of the GCLS demonstrates robust psychometric properties and can be recommended for use in clinical practice and research with German-speaking transgender and gender diverse populations.

\newpage

\section{References}\label{references}

Beaton, D. E., Bombardier, C., Guillemin, F., \& Ferraz, M. B. (2000). Guidelines for the process of cross-cultural adaptation of self-report measures. \emph{Spine, 25}(24), 3186-3191.

\citet{article}\{jones2019,
title=\{The Gender Congruence and Life Satisfaction Scale (GCLS): Development and validation of a scale to measure outcomes from transgender health services\},
author=\{Jones, Bethany A and Bouman, Walter Pierre and Haycraft, Emma and Arcelus, Jon\},
journal=\{International Journal of Transgenderism\},
volume=\{20\},
number=\{1\},
pages=\{63--80\},
year=\{2019\},
publisher=\{Taylor \& Francis\},
doi=\{10.1080/15532739.2018.1453425\}
\}

{[}Additional references to be added{]}


\bibliography{../../references/references.bib}

\end{document}
