\documentclass[man,floatsintext,12pt]{apa7}

% Essential packages
\usepackage[utf8]{inputenc}
\usepackage[T1]{fontenc}
\usepackage{graphicx}
\usepackage{booktabs}
\usepackage{caption}
\usepackage{subcaption}
\usepackage{float}
\usepackage{csquotes}
\usepackage[style=apa,sortcites=true,sorting=nyt,backend=biber]{biblatex}

% Document metadata
\title{Validation of the German Version of the Gender Congruence and
Life Satisfaction Scale (G-GCLS)}
\author{Jan Ben Schulze\textsuperscript{1}* \and Flavio
Ammann\textsuperscript{1} \and Bethany A.
Jones\textsuperscript{2,3} \and Roland von
Känel\textsuperscript{1} \and Sebastian Euler\textsuperscript{1}}
\affiliation{Department of Consultation-Liaison Psychiatry and
Psychosomatic Medicine, University Hospital Zurich \& University of
Zurich \and Department of Consultation-Liaison Psychiatry and
Psychosomatic Medicine, University Hospital Zurich \& University of
Zurich \and Nottingham Centre for Transgender Health; School of Sport,
Exercise and Health Sciences, Loughborough University \and Department of
Consultation-Liaison Psychiatry and Psychosomatic Medicine, University
Hospital Zurich \& University of Zurich \and Department of
Consultation-Liaison Psychiatry and Psychosomatic Medicine, University
Hospital Zurich \& University of Zurich}
\leftheader{Running head: GCLS GERMAN VALIDATION}

\abstract{The Gender Congruence and Life Satisfaction Scale (GCLS) is a
validated measure assessing outcomes in transgender and gender diverse
individuals. This study presents the translation and validation of the
German version of the GCLS. Using a sample of 293 participants, we
conducted exploratory factor analysis and assessed psychometric
properties. The German GCLS demonstrated excellent internal consistency
(Cronbach's α ranging from .78 to .90) and replicated the seven-factor
structure of the original scale. Factor analysis revealed strong
correspondence with the original English version, particularly for the
Chest, Genitalia, and Social Gender Role Recognition subscales. The
results support the German GCLS as a reliable and valid instrument for
assessing gender congruence and life satisfaction in German-speaking
transgender and gender diverse populations.}

\keywords{gender congruence; life satisfaction; scale validation;
transgender health; psychometrics}

\begin{document}

\textsuperscript{1} Department of Consultation-Liaison Psychiatry and
Psychosomatic Medicine, University Hospital Zurich \& University of
Zurich, Zurich, Switzerland\\
\textsuperscript{2} Nottingham Centre for Transgender Health,
Nottingham, UK\\
\textsuperscript{3} School of Sport, Exercise and Health Sciences,
Loughborough University, Loughborough, UK

* Corresponding author:
\href{mailto:jan.schulze@usz.ch}{\nolinkurl{jan.schulze@usz.ch}}

\section{Ethics Statement}\label{ethics-statement}

The study underwent review by the Cantonal Ethics Committee of the
Canton of Zurich, Switzerland (BASEC No.~Req-2022-00630). The Committee
confirmed that the planned anonymous survey does not fall within the
scope of the Human Research Act and, therefore, does not require its
approval to conduct. The authors affirm that all procedures involved in
this study comply with the ethical standards of the relevant national
and institutional committees on human experimentation and the 1975
Declaration of Helsinki, as amended in 2008.

\section{Introduction}\label{introduction}

The provision of healthcare for transgender individuals within the
German-speaking healthcare system is confronting existential challenges.
Transgender people experience a fundamental incongruence between the sex
assigned at birth and their lived gender identity, often identifying as
gender-neutral, non-gender, or gender-queer
\citep{arcelus2017, richards2016}. This discrepancy is frequently
accompanied by significant distress, necessitating specialized and
multidisciplinary treatment approaches \citep{beek2015}.

Despite substantial progress achieved through gender-affirming
interventions---which numerous studies document as yielding notable
improvements in mental health and overall quality of life
\citep{coleman2012, wylie2014, dhejne2016, jones2016, marshall2016}---the
evaluation of treatment outcomes has predominantly focused on singular
aspects. This unidimensional approach fails to capture the complex,
multifaceted needs of transgender individuals
\citep{bouman2017, heylens2014, murad2010, witcomb2018}.

Moreover, the established measurement instruments---for instance, the
Utrecht Gender Dysphoria Scale \citep{cohen1997} and the Hamburg Body
Drawing Scale \citep{becker2016}---are generally based on a binary
understanding of gender, thereby inadequately reflecting the growing
diversity of transgender self-definitions \citep{beek2015, clarke2018}.
In addition, the use of multiple questionnaires imposes a significant
burden on respondents, which can detrimentally affect both participation
rates and the validity and reliability of the collected data
\citep{rolstad2011, turner2007}.

A parallel methodological gap in the German-speaking region further
exacerbates these issues. Although representative surveys are still
lacking, estimates suggest that approximately 0.6-0.7\% of the
population may be transgender \citep{beek2015, cohen1997}. Without
robust, representative data, key questions regarding need-based,
patient-centered care---such as the appropriateness of treatment
offerings and long-term care requirements---remain inadequately
addressed.

The Gender Congruence and Life Satisfaction Scale (GCLS) was originally
developed and validated by Jones et al. \citeyearpar{jones2019} as a
comprehensive measure addressing these limitations. While the original
English version has demonstrated robust psychometric properties with a
seven-factor structure, showing good reliability and validity across
different gender groups, there is a pressing need for validated
translations to facilitate cross-cultural research and clinical
applications in German-speaking contexts.

This study aims to:

\begin{enumerate}
\def\labelenumi{\arabic{enumi}.}
\tightlist
\item
  Develop a German translation of the GCLS following rigorous
  translation procedures
\item
  Validate the German version in a sample of German-speaking transgender
  and gender diverse individuals
\item
  Examine the psychometric properties and factor structure in comparison
  to the original English version
\item
  Establish the reliability and validity of the German GCLS for clinical
  and research applications
\end{enumerate}

\section{Methods}\label{methods}

\subsection{Translation Process}\label{translation-process}

The translation process followed established guidelines for
cross-cultural adaptation (Beaton et al., 2000). This included:

\begin{enumerate}
\def\labelenumi{\arabic{enumi}.}
\tightlist
\item
  Forward translation by two independent bilingual translators
\item
  Synthesis of translations
\item
  Back-translation by two independent translators
\item
  Expert committee review
\item
  Pilot testing with cognitive interviews
\item
  Final revision
\end{enumerate}

\subsection{Participants}\label{participants}

The validation sample consisted of 293 German-speaking transgender and
gender diverse individuals recruited through gender identity clinics and
community organizations. Participants completed the German GCLS as part
of a larger survey battery.

\subsection{Measures}\label{measures}

The GCLS consists of 38 items across seven subscales:

\begin{longtable}[]{@{}
  >{\raggedright\arraybackslash}p{(\linewidth - 4\tabcolsep) * \real{0.3387}}
  >{\raggedright\arraybackslash}p{(\linewidth - 4\tabcolsep) * \real{0.2581}}
  >{\raggedright\arraybackslash}p{(\linewidth - 4\tabcolsep) * \real{0.4032}}@{}}
\caption{GCLS Subscales and Item Distribution}\tabularnewline
\toprule\noalign{}
\begin{minipage}[b]{\linewidth}\raggedright
Subscale
\end{minipage} & \begin{minipage}[b]{\linewidth}\raggedright
Items
\end{minipage} & \begin{minipage}[b]{\linewidth}\raggedright
Description
\end{minipage} \\
\midrule\noalign{}
\endfirsthead
\toprule\noalign{}
\begin{minipage}[b]{\linewidth}\raggedright
Subscale
\end{minipage} & \begin{minipage}[b]{\linewidth}\raggedright
Items
\end{minipage} & \begin{minipage}[b]{\linewidth}\raggedright
Description
\end{minipage} \\
\midrule\noalign{}
\endhead
\bottomrule\noalign{}
\endlastfoot
Genitalia (GEN) & 14, 21, 25, 26, 27, 29 & Satisfaction with genital
characteristics \\
Chest (CH) & 15, 18, 28, 30 & Satisfaction with chest appearance \\
Other Secondary Sex Characteristics (SSC) & 17, 23, 24 & Satisfaction
with secondary sex characteristics \\
Social Gender Role Recognition (SGR) & 16, 19, 20, 22 & Experience of
social recognition in gender role \\
Physical and Emotional Intimacy (PEI) & 3, 5, 32, 33 & Satisfaction with
physical and emotional intimacy \\
Psychological Functioning (PF) & 1, 2, 4, 6, 7, 8, 9, 11, 12, 13 &
General psychological well-being \\
Life Satisfaction (LS) & 10, 31, 34, 35, 36, 37, 38 & Overall life
satisfaction \\
\end{longtable}

\subsection{Statistical Analysis}\label{statistical-analysis}

Our analytical approach followed established guidelines for
cross-cultural validation of psychological instruments
\citep{beaton2000}. For the factor analysis, we specifically considered
recommendations for scale validation in transgender health research
\citep{jones2019}.

\subsubsection{Factor Analysis Strategy}\label{factor-analysis-strategy}

We conducted an exploratory factor analysis (EFA) rather than a
confirmatory factor analysis (CFA) in this first validation of the
German version. This approach allows us to examine whether the factor
structure emerges naturally in the German context, rather than imposing
the original structure. This is particularly important given potential
cultural differences in the experience and expression of gender
congruence.

The analysis proceeded in several steps:

\begin{enumerate}
\def\labelenumi{\arabic{enumi}.}
\tightlist
\item
  \textbf{Data Screening and Preparation}

  \begin{itemize}
  \tightlist
  \item
    Assessment of sampling adequacy using Kaiser-Meyer-Olkin (KMO) test
  \item
    Bartlett's test of sphericity to confirm data suitability
  \item
    Examination of item distributions and correlations
  \end{itemize}
\item
  \textbf{Factor Extraction}

  \begin{itemize}
  \tightlist
  \item
    Principal axis factoring as the extraction method
  \item
    Determination of factor number using multiple criteria:

    \begin{itemize}
    \tightlist
    \item
      Kaiser criterion (eigenvalues \textgreater{} 1)
    \item
      Scree plot examination
    \item
      Parallel analysis
    \item
      Theoretical considerations from the original scale
    \end{itemize}
  \end{itemize}
\item
  \textbf{Rotation Method}

  \begin{itemize}
  \tightlist
  \item
    Varimax rotation to maintain consistency with the original
    validation
  \item
    This orthogonal rotation method maximizes variance of loadings
    within factors
  \item
    Facilitates comparison with the original English version's factor
    structure
  \end{itemize}
\item
  \textbf{Factor Loading Evaluation}

  \begin{itemize}
  \tightlist
  \item
    Primary loadings ≥ 0.40 considered significant
  \item
    Cross-loadings examined when \textgreater{} 0.30
  \item
    Items assigned to factors based on highest loading
  \item
    Special attention to items that loaded differently from original
    version
  \end{itemize}
\end{enumerate}

\subsubsection{Additional Psychometric
Analyses}\label{additional-psychometric-analyses}

To complement the factor analysis, we conducted:

\begin{enumerate}
\def\labelenumi{\arabic{enumi}.}
\tightlist
\item
  Internal consistency analysis (Cronbach's α) for each subscale
\item
  Item-total correlations to assess item performance
\item
  Inter-scale correlations to examine discriminant validity
\item
  Comparison with original English version factor structure
\end{enumerate}

This comprehensive approach allows us to evaluate both the structural
validity of the German translation and its comparability with the
original instrument, while remaining sensitive to potential cultural
differences in the assessment of gender congruence and life
satisfaction.

\section{Results}\label{results}

\subsection{Factor Structure}\label{factor-structure}

The exploratory factor analysis revealed a seven-factor solution
explaining 57.1\% of the total variance. The Kaiser-Meyer-Olkin measure
(KMO = 0.920) indicated excellent sampling adequacy. The analysis
revealed the following factor structure:

\begin{enumerate}
\def\labelenumi{\arabic{enumi}.}
\tightlist
\item
  \textbf{Factor 1: Psychological Functioning and Life Satisfaction}
  (19.1\% variance)

  \begin{itemize}
  \tightlist
  \item
    16 items with primary loadings \textgreater{} 0.40
  \item
    Strongest loadings from items related to psychological well-being
  \item
    Items: 1-7, 9-13, 31, 32, 34, 37, 38
  \end{itemize}
\item
  \textbf{Factor 2: Genitalia} (10.5\% variance)

  \begin{itemize}
  \tightlist
  \item
    6 items with strong loadings (0.41-0.87)
  \item
    Items: 14, 21, 25, 26, 27, 29
  \end{itemize}
\item
  \textbf{Factor 3: Chest} (8.2\% variance)

  \begin{itemize}
  \tightlist
  \item
    4 items with high loadings (0.62-0.84)
  \item
    Items: 15, 18, 28, 30
  \end{itemize}
\item
  \textbf{Factor 4: Social Gender Role} (6.9\% variance)

  \begin{itemize}
  \tightlist
  \item
    4 items with strong loadings (0.56-0.74)
  \item
    Items: 16, 19, 20, 22
  \end{itemize}
\item
  \textbf{Factor 5: Secondary Sex Characteristics} (4.8\% variance)

  \begin{itemize}
  \tightlist
  \item
    3 items with moderate to strong loadings (0.50-0.77)
  \item
    Items: 17, 23, 24
  \end{itemize}
\item
  \textbf{Factor 6: Mixed Physical and Emotional Aspects} (4.0\%
  variance)

  \begin{itemize}
  \tightlist
  \item
    3 items with moderate loadings
  \item
    Items: 33, 35, 36
  \end{itemize}
\item
  \textbf{Factor 7: Residual Factor} (3.7\% variance)

  \begin{itemize}
  \tightlist
  \item
    2 items with significant loadings
  \item
    Items: 8, 13
  \end{itemize}
\end{enumerate}

The eigenvalues for the seven factors were 13.10, 3.69, 2.42, 1.61,
1.33, 1.20, and 1.16, respectively.

\subsection{Internal Consistency}\label{internal-consistency}

All subscales demonstrated good to excellent internal consistency:

\begin{longtable}[]{@{}lrl@{}}
\caption{Internal Consistency and Item-Total
Correlations}\tabularnewline
\toprule\noalign{}
Subscale & Cronbach.s.α & Item.Total.Correlation.Range \\
\midrule\noalign{}
\endfirsthead
\toprule\noalign{}
Subscale & Cronbach.s.α & Item.Total.Correlation.Range \\
\midrule\noalign{}
\endhead
\bottomrule\noalign{}
\endlastfoot
GEN & 0.883 & 0.397-0.838 \\
CH & 0.900 & 0.684-0.854 \\
SSC & 0.795 & 0.535-0.734 \\
SGR & 0.842 & 0.632-0.702 \\
PEI & 0.779 & 0.513-0.667 \\
PF & 0.884 & 0.338-0.750 \\
LS & 0.811 & 0.321-0.684 \\
\end{longtable}

\subsection{Convergence with Original
Version}\label{convergence-with-original-version}

Factor correlation analysis revealed strong correspondence between the
German and English versions, particularly for:

\begin{enumerate}
\def\labelenumi{\arabic{enumi}.}
\tightlist
\item
  Chest subscale (r = 0.694)
\item
  Secondary Sex Characteristics subscale (r = 0.687)
\item
  Genitalia subscale (r = 0.560)
\end{enumerate}

\section{Discussion}\label{discussion}

The German version of the GCLS demonstrates strong psychometric
properties and replicates the factor structure of the original English
version. The high internal consistency values (α \textgreater{} 0.77 for
all subscales) indicate reliable measurement of the intended constructs.
The factor analysis results support the theoretical seven-factor
structure, although with some differences in factor loading patterns.

\subsection{Strengths and Limitations}\label{strengths-and-limitations}

Strengths: - Large validation sample - Comprehensive psychometric
evaluation - Strong reliability coefficients - Clear factor structure

Limitations: - Cross-sectional design - Lack of test-retest reliability
data - Need for confirmatory factor analysis in an independent sample

\section{Conclusion}\label{conclusion}

The German version of the GCLS demonstrates robust psychometric
properties and can be recommended for use in clinical practice and
research with German-speaking transgender and gender diverse
populations. Future research should focus on:

\begin{enumerate}
\def\labelenumi{\arabic{enumi}.}
\tightlist
\item
  Confirmatory factor analysis
\item
  Test-retest reliability
\item
  Measurement invariance testing
\item
  External validity assessment
\end{enumerate}

\section{References}\label{references}

Beaton, D. E., Bombardier, C., Guillemin, F., \& Ferraz, M. B. (2000).
Guidelines for the process of cross-cultural adaptation of self-report
measures. \emph{Spine, 25}(24), 3186-3191.

\citet{article}\{jones2019, title=\{The Gender Congruence and Life
Satisfaction Scale (GCLS): Development and validation of a scale to
measure outcomes from transgender health services\}, author=\{Jones,
Bethany A and Bouman, Walter Pierre and Haycraft, Emma and Arcelus,
Jon\}, journal=\{International Journal of Transgenderism\},
volume=\{20\}, number=\{1\}, pages=\{63--80\}, year=\{2019\},
publisher=\{Taylor \& Francis\}, doi=\{10.1080/15532739.2018.1453425\}
\}

{[}Additional references to be added{]}

\end{document} 