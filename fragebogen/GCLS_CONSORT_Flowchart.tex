\documentclass[11pt,a4paper]{article}
\usepackage[utf8]{inputenc}
\usepackage[T1]{fontenc}
\usepackage[ngerman]{babel}
\usepackage{geometry}
\geometry{a4paper, margin=1.5cm}
\usepackage{tikz}
\usetikzlibrary{shapes.geometric, arrows, positioning, fit, calc}
\usepackage{xcolor}
\usepackage{helvet}
\renewcommand{\familydefault}{\sfdefault}

% Farben (CONSORT-Style)
\definecolor{darkblue}{RGB}{0,51,102}
\definecolor{lightblue}{RGB}{173,216,230}
\definecolor{lightgray}{RGB}{240,240,240}
\definecolor{darkgreen}{RGB}{0,100,0}
\definecolor{orange}{RGB}{255,165,0}
\definecolor{red}{RGB}{220,20,60}

% TikZ Styles
\tikzstyle{recruitment} = [rectangle, minimum width=4cm, minimum height=1.2cm, text centered, text width=4cm, draw=darkblue, fill=lightblue, font=\small\bfseries]
\tikzstyle{process} = [rectangle, minimum width=4cm, minimum height=1cm, text centered, text width=4cm, draw=darkblue, fill=lightgray, font=\small]
\tikzstyle{excluded} = [rectangle, minimum width=3cm, minimum height=0.8cm, text centered, text width=3cm, draw=red, fill=red!20, font=\footnotesize]
\tikzstyle{final} = [rectangle, minimum width=4cm, minimum height=1cm, text centered, text width=4cm, draw=darkgreen, fill=green!20, font=\small\bfseries]
\tikzstyle{arrow} = [thick,->,>=stealth, color=darkblue]

\begin{document}

\title{
    {\Large\bfseries\color{darkblue} GCLS-Gv1.1 Validierungsstudie}\\[0.2cm]
    {\large CONSORT-Flowchart der Rekrutierung}
}
\author{}
\date{}
\maketitle

\vspace{-1cm}

\begin{center}
\begin{tikzpicture}[node distance=1.3cm]

% INITIAL POPULATION
\node (initial) [recruitment] {
    \textbf{Transgender Personen in Datenbank}\\
    \textbf{(ICD-10: F64, 2001-2021)}\\
    \textbf{N = 1161}
};

% TEST RUN
\node (testrun) [process, below of=initial] {
    \textbf{Testlauf}\\
    N = 15\\
    Fragebogen-Anpassung
};

% FIRST CONTACT - LETTER
\node (letter) [recruitment, below of=testrun] {
    \textbf{1. Kontakt: Postbrief}\\
    \textbf{09.01.2023}\\
    Kontaktiert: n = 1146\\
    Unzustellbar: n = 332
};

% FIRST WAVE RESPONSE
\node (resp1) [process, below of=letter] {
    \textbf{1. Welle Antworten}\\
    \textbf{10.01.2023: n = 52}\\
    Kumulativ: n = 52 (4.5\%)
};

% SECOND CONTACT - EMAIL
\node (email) [recruitment, below of=resp1] {
    \textbf{2. Kontakt: E-Mail}\\
    \textbf{29.01.2023}\\
    (20 Tage nach Postbrief)\\
    Unzustellbar: n = 208
};

% SECOND WAVE RESPONSE
\node (resp2) [process, below of=email] {
    \textbf{2. Welle Antworten}\\
    \textbf{29-30.01.2023: n = 59}\\
    Kumulativ: n = 111 (9.7\%)
};

% THIRD CONTACT - SMS
\node (sms) [recruitment, below of=resp2] {
    \textbf{3. Kontakt: SMS}\\
    \textbf{08.02.2023}\\
    (30 Tage nach Postbrief)\\
    Unzustellbar: n = 189
};

% THIRD WAVE RESPONSE
\node (resp3) [process, below of=sms] {
    \textbf{3. Welle Antworten}\\
    \textbf{09.02.2023: n = 42}\\
    Kumulativ: n = 153 (13.2\%)
};

% WEITERE ANTWORTEN
\node (additional) [process, below of=resp3] {
    \textbf{Weitere Antworten}\\
    \textbf{bis 15.02.2023}\\
    n = 140 (über 37 Tage verteilt)
};

% FINAL SAMPLE
\node (final) [final, below of=additional] {
    \textbf{Finale Stichprobe}\\
    \textbf{N = 293 (25.3\%)}\\
    Alter: 18-75 Jahre (M = 39.8, SD = 16.4)
};

% EXCLUSIONS (right side)
\node (excl1) [excluded, right of=letter, xshift=4cm] {
    Ausgeschlossen:\\
    Unzustellbare Briefe\\
    n = 332
};

\node (excl2) [excluded, right of=email, xshift=4cm] {
    Ausgeschlossen:\\
    Unzustellbare E-Mails\\
    n = 208
};

\node (excl3) [excluded, right of=sms, xshift=4cm] {
    Ausgeschlossen:\\
    Unzustellbare SMS\\
    n = 189
};

\node (excl4) [excluded, right of=final, xshift=4cm] {
    Keine Teilnahme:\\
    n = 868 (74.7\%)\\
    Gründe unbekannt
};

% ARROWS
\draw [arrow] (initial) -- (testrun);
\draw [arrow] (testrun) -- (letter);
\draw [arrow] (letter) -- (resp1);
\draw [arrow] (resp1) -- (email);
\draw [arrow] (email) -- (resp2);
\draw [arrow] (resp2) -- (sms);
\draw [arrow] (sms) -- (resp3);
\draw [arrow] (resp3) -- (additional);
\draw [arrow] (additional) -- (final);

% EXCLUSION ARROWS
\draw [arrow, color=red] (letter) -- (excl1);
\draw [arrow, color=red] (email) -- (excl2);
\draw [arrow, color=red] (sms) -- (excl3);
\draw [arrow, color=red] (final) -- (excl4);

% TIMELINE ANNOTATIONS
\node[font=\footnotesize, color=darkblue, rotate=90] at ($(initial) + (-3,0)$) {Datenbank-Identifizierung};
\node[font=\footnotesize, color=darkblue, rotate=90] at ($(letter) + (-3,0)$) {Rekrutierungsphase};
\node[font=\footnotesize, color=darkblue, rotate=90] at ($(final) + (-3,0)$) {Datenerhebung};

\end{tikzpicture}
\end{center}

\vspace{1cm}

% REKRUTIERUNGS-EFFEKTIVITÄT
\begin{center}
\begin{tabular}{|l|c|c|c|c|}
\hline
\textbf{Kontakt-Methode} & \textbf{Datum} & \textbf{Peak-Antworten} & \textbf{Kumulativ} & \textbf{Effektivität} \\
\hline
1. Postbrief & 09.01.2023 & 52 (10.01.) & 52 & 4.5\% \\
2. E-Mail & 29.01.2023 & 59 (29-30.01.) & 111 & 9.7\% \\
3. SMS & 08.02.2023 & 42 (09.02.) & 153 & 13.2\% \\
\hline
\textbf{Gesamt} & \textbf{37 Tage} & \textbf{293} & \textbf{293} & \textbf{25.3\%} \\
\hline
\end{tabular}
\end{center}

\vspace{0.5cm}

\begin{center}
\footnotesize
\textbf{Studiendetails:} Querschnitt-Validierungsstudie | BASEC Nr. Req-2022-00630\\
UniversitätsSpital Zürich | Klinik für Konsiliar- und Liaisons-Psychiatrie
\end{center}

\end{document} 