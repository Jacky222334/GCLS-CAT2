\documentclass[11pt,a4paper]{article}
\usepackage[utf8]{inputenc}
\usepackage[T1]{fontenc}
\usepackage[ngerman]{babel}
\usepackage{geometry}
\geometry{a4paper, margin=2cm}
\usepackage{tikz}
\usetikzlibrary{shapes.geometric, arrows, positioning}
\usepackage{xcolor}

\definecolor{darkblue}{RGB}{0,51,102}
\definecolor{lightblue}{RGB}{173,216,230}
\definecolor{lightgray}{RGB}{240,240,240}

\tikzstyle{box} = [rectangle, minimum width=3cm, minimum height=1cm, text centered, draw=darkblue, fill=lightblue]
\tikzstyle{arrow} = [thick,->,>=stealth]

\begin{document}

\title{\textbf{GCLS-Gv1.1 Validierungsstudie\\Studienablauf-Flowchart}}
\author{}
\date{}
\maketitle

\begin{center}
\begin{tikzpicture}[node distance=2cm]

\node (start) [box] {Studienstart\\Ethikantrag genehmigt};
\node (recruit) [box, below of=start] {Rekrutierung\\n = 350 angesprochen};
\node (screen) [box, below of=recruit] {Screening\\Einschlusskriterien};
\node (consent) [box, below of=screen] {Informierte Einwilligung\\n = 293};
\node (data) [box, below of=consent] {Datenerhebung\\GCLS-Gv1.1 (38 Items)};
\node (analysis) [box, below of=data] {Statistische Analyse\\Faktorenanalyse};
\node (results) [box, below of=analysis] {Ergebnisse\\7-Faktoren-Struktur};
\node (publish) [box, below of=results] {Manuskript\\Publikation};

\draw [arrow] (start) -- (recruit);
\draw [arrow] (recruit) -- (screen);
\draw [arrow] (screen) -- (consent);
\draw [arrow] (consent) -- (data);
\draw [arrow] (data) -- (analysis);
\draw [arrow] (analysis) -- (results);
\draw [arrow] (results) -- (publish);

\node[right of=screen, xshift=3cm] {Ausgeschlossen: n = 57};

\end{tikzpicture}
\end{center}

\vspace{1cm}

\textbf{Studiendetails:}
\begin{itemize}
\item Querschnitt-Validierungsstudie
\item N = 293 transgender Personen
\item Alter: 18-75 Jahre (M = 39.8, SD = 16.4)
\item BASEC Nr. Req-2022-00630
\end{itemize}

\end{document} 