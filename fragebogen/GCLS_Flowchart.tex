\documentclass[11pt,a4paper]{article}

% ===========================================
% PAKETE
% ===========================================
\usepackage[utf8]{inputenc}
\usepackage[T1]{fontenc}
\usepackage[ngerman]{babel}
\usepackage{geometry}
\geometry{a4paper, margin=2cm}

% TikZ für Flowchart
\usepackage{tikz}
\usetikzlibrary{shapes.geometric, arrows, positioning, fit, calc}

% Farben
\usepackage{xcolor}
\definecolor{darkblue}{RGB}{0,51,102}
\definecolor{lightblue}{RGB}{173,216,230}
\definecolor{lightgray}{RGB}{240,240,240}
\definecolor{mediumgray}{RGB}{128,128,128}
\definecolor{darkgreen}{RGB}{0,100,0}
\definecolor{orange}{RGB}{255,165,0}

% Schriften
\usepackage{helvet}
\renewcommand{\familydefault}{\sfdefault}

% ===========================================
% TIKZ STYLES
% ===========================================
\tikzstyle{startstop} = [rectangle, rounded corners, minimum width=3cm, minimum height=1cm, text centered, draw=darkblue, fill=lightblue, font=\bfseries]

\tikzstyle{process} = [rectangle, minimum width=4cm, minimum height=1cm, text centered, draw=darkblue, fill=lightgray, font=\small]

\tikzstyle{decision} = [diamond, minimum width=3cm, minimum height=1cm, text centered, draw=orange, fill=orange!20, font=\small]

\tikzstyle{data} = [trapezium, trapezium left angle=70, trapezium right angle=110, minimum width=3cm, minimum height=1cm, text centered, draw=darkgreen, fill=green!20, font=\small]

\tikzstyle{arrow} = [thick,->,>=stealth, color=darkblue]

\tikzstyle{exclusion} = [rectangle, minimum width=2.5cm, minimum height=0.8cm, text centered, draw=red, fill=red!20, font=\footnotesize]

% ===========================================
% DOKUMENT
% ===========================================
\begin{document}

\title{
    {\Large\bfseries\color{darkblue} GCLS-Gv1.1 Validierungsstudie}\\[0.2cm]
    {\large Studienablauf-Flowchart}
}
\author{}
\date{}
\maketitle

\vspace{-1cm}

\begin{center}
\begin{tikzpicture}[node distance=1.5cm]

% START
\node (start) [startstop] {Studienstart\\Ethikantrag genehmigt};

% REKRUTIERUNG
\node (recruit) [process, below of=start] {Rekrutierung\\Transgender Personen\\(n = 350 angesprochen)};

% EINSCHLUSSKRITERIEN
\node (inclusion) [decision, below of=recruit, yshift=-0.5cm] {Einschlusskriterien\\erfüllt?};

% AUSSCHLUSS
\node (exclude) [exclusion, right of=inclusion, xshift=3cm] {Ausgeschlossen\\(n = 57)\\- Alter < 18\\- Unvollständige Daten\\- Keine Einwilligung};

% INFORMIERTE EINWILLIGUNG
\node (consent) [process, below of=inclusion, yshift=-0.5cm] {Informierte Einwilligung\\und Studienaufklärung\\(n = 293)};

% DATENERHEBUNG
\node (datacoll) [data, below of=consent] {Datenerhebung\\- GCLS-Gv1.1 (38 Items)\\- Demographische Daten\\- Zusatzfragebögen};

% DATENVERARBEITUNG
\node (dataproc) [process, below of=datacoll] {Datenverarbeitung\\- Qualitätskontrolle\\- Missing Data Analysis\\- Datenbereinigung};

% DATENQUALITÄT
\node (quality) [decision, below of=dataproc, yshift=-0.5cm] {Datenqualität\\ausreichend?};

% NACHERHEBUNG
\node (recheck) [process, right of=quality, xshift=3cm] {Nacherhebung\\bei n = 12\\Teilnehmenden};

% HAUPTANALYSE
\node (analysis) [process, below of=quality, yshift=-0.5cm] {Statistische Analyse\\- Explorative Faktorenanalyse\\- Interne Konsistenz\\- Validitätsprüfung};

% ERGEBNISSE
\node (results) [data, below of=analysis] {Ergebnisse\\- 7-Faktoren-Struktur\\- Cronbachs α = .78-.90\\- Gute Modellpassung};

% INTERPRETATION
\node (interpret) [process, below of=results] {Interpretation\\und Diskussion\\der Ergebnisse};

% PUBLIKATION
\node (publish) [startstop, below of=interpret] {Manuskript\\und Publikation};

% PFEILE
\draw [arrow] (start) -- (recruit);
\draw [arrow] (recruit) -- (inclusion);
\draw [arrow] (inclusion) -- node[anchor=west] {Ja} (consent);
\draw [arrow] (inclusion) -- node[anchor=south] {Nein} (exclude);
\draw [arrow] (consent) -- (datacoll);
\draw [arrow] (datacoll) -- (dataproc);
\draw [arrow] (dataproc) -- (quality);
\draw [arrow] (quality) -- node[anchor=west] {Ja} (analysis);
\draw [arrow] (quality) -- node[anchor=south] {Nein} (recheck);
\draw [arrow] (recheck) |- (dataproc);
\draw [arrow] (analysis) -- (results);
\draw [arrow] (results) -- (interpret);
\draw [arrow] (interpret) -- (publish);

% SAMPLE SIZE ANNOTATIONS
\node[font=\footnotesize, color=mediumgray] at ($(consent) + (2.5,0)$) {n = 293};
\node[font=\footnotesize, color=mediumgray] at ($(analysis) + (2.5,0)$) {Final Sample\\n = 293};

\end{tikzpicture}
\end{center}

\vspace{1cm}

% LEGENDE
\begin{center}
\begin{tikzpicture}[node distance=0.5cm]
\node[startstop, scale=0.7] (leg1) {Start/Ende};
\node[process, scale=0.7, right of=leg1, xshift=2cm] (leg2) {Prozess};
\node[decision, scale=0.7, right of=leg2, xshift=2cm] (leg3) {Entscheidung};
\node[data, scale=0.7, right of=leg3, xshift=2cm] (leg4) {Daten};
\node[exclusion, scale=0.7, right of=leg4, xshift=2cm] (leg5) {Ausschluss};

\node[font=\footnotesize, below of=leg1] {Start/Ende};
\node[font=\footnotesize, below of=leg2] {Prozess};
\node[font=\footnotesize, below of=leg3] {Entscheidung};
\node[font=\footnotesize, below of=leg4] {Daten};
\node[font=\footnotesize, below of=leg5] {Ausschluss};
\end{tikzpicture}
\end{center}

\vspace{1cm}

% STUDIENDETAILS
\begin{center}
\begin{tabular}{ll}
\textbf{Studientyp:} & Querschnitt-Validierungsstudie \\
\textbf{Stichprobe:} & N = 293 transgender Personen \\
\textbf{Altersbereich:} & 18-75 Jahre (M = 39.8, SD = 16.4) \\
\textbf{Rekrutierung:} & Online-Plattformen, Selbsthilfegruppen \\
\textbf{Studienzeit:} & 6 Monate (2022-2023) \\
\textbf{Ethik:} & BASEC Nr. Req-2022-00630 \\
\end{tabular}
\end{center}

\vspace{0.5cm}

\begin{center}
\footnotesize
\textbf{GCLS-Gv1.1 Validierungsstudie} | UniversitätsSpital Zürich\\
Klinik für Konsiliar- und Liaisons-Psychiatrie sowie Psychosomatische Medizin
\end{center}

\end{document} 